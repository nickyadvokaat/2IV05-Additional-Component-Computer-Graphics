\section{Analysis}
In this section we will describe the problems we will have to deal with to create \name. These problems can be roughly divided between: The physics of the bridge builder, the collapsing of the bridge and the interface of the bridge builder.
\subsection{Physics}
The problem with the physics is that we have to determine whether the bridge will collapse or not. This depends on the gravity and on the upward force. We can simply compute the gravity by taking the sum of the mass of each building block. The upward force is determined by pulling robes, and blocks that are build on land.  If we could keep it very simple: if the upward force is bigger than the gravity on the bridge, the bridge won't collapse. We will also have to deal with other factors. If all upward forces are at one side of the bridge, the bridge should collapse too. 
\subsection{Collapsing of the Bridge}
Visualizing the collapsing of the bridge is very difficult and depends a bit on the physics. Some buiding blocks fall faster and some could break. Most building blocks will rotate and there will be a lot of collisionss between the building blocks. We also have to visualize how the train falls down, and determine if it stays intact or some wagons break off. There could also be some special effect like smoke and fire. 
\subsection{Interface}
There are some problems that have to be fixed, to make the interface simple and easy to use. The action of connecting two building blocks should be easy, so the user does not have to click the exact pixel to make the intended connection. \\
It should be easy for the user to rotate the building blocks in the correct direction. \\
