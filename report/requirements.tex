\section{Requirements}
For  \name, we will describe three different kinds of requirements. The functionality, interaction and presentation requirements. The functionality requirements describe which functions should be possible in \name. The interaction requirements describe how the user should be able to interact with the system. The presentation requirements describe what \name \ should look like. 
 \subsection{Functionality}
 The functionality is divided in some different functions and the requirements of this functions. Also we have some functionality that should work, with all functions.
 The following functionality should be in the \name, to make it useful.
  \begin{itemize}
  \item There should be at least 2 different kinds of building blocks. 
  \item There should be some basic physics.
  \end{itemize}
  There should be default levels, for the user to solve. In this default levels, the users has a standard amount of building blocks, to make the bridge strong enough for a train to ride over it. 
 \begin{itemize}
 \item The user can only use the building blocks defined for the level
 \item It should be possible to make a bridge strong enough for a train to ride over it
 \end{itemize}
  It should be possible for the user to create their own levels, so that other users can try to solve this level. 
 \begin{itemize}
 \item The user has to solve the level himself, before other users can try to solve the level.
 \item Other users can only use the building blocks, that the user used to make a bridge.
 \item The user should be able to change the properties of building blocks, for example they should be able to change the height, width, color and strength. Only the creator of the level can change this properties. The other player that tries to solve the level cannot change this properties.
 \end{itemize}

\subsection{Interaction}
The interaction should be simple and easy. It should be easy to connect two building blocks. This means that the user does not have to set the connection precisely, but that the building blocks are connected by \name. It should be simple to select the correct building block and the camere should work simple.


\subsection{Presentation}
The presentation will be a basic 3D world. The basic 3D world will be a simple cliff. There should be a tool bar on the right to select building blocks.
Additional requirements for the presentation:
 \begin{itemize}
 \item Better looking building blocks
 \item More than one 3D world to build bridges in. This could be a over a canal or a river. This could be in different places, like in a city or a mountain.
 \item Customizable 3D world, so that users can change the depth, the width etc. of the ravine. So that users can add rocks to build on in the middle of the ravine.
 \end{itemize}